% last updated in April 2002 by Antje Endemann
% Based on CVPR 07 and LNCS, with modifications by DAF, AZ and elle, 2008 and AA, 2010, and CC, 2011

\documentclass[runningheads]{llncs}
\usepackage{graphicx}
\usepackage{amsmath,amssymb} % define this before the line numbering.
\usepackage{ruler}
\usepackage{color}
\usepackage{url}
\usepackage[width=122mm,left=12mm,paperwidth=146mm,height=193mm,top=12mm,paperheight=217mm]{geometry}
\begin{document}
% \renewcommand\thelinenumber{\color[rgb]{0.2,0.5,0.8}\normalfont\sffamily\scriptsize\arabic{linenumber}\color[rgb]{0,0,0}}
% \renewcommand\makeLineNumber {\hss\thelinenumber\ \hspace{6mm} \rlap{\hskip\textwidth\ \hspace{6.5mm}\thelinenumber}}
% \linenumbers \
\pagestyle{headings}
\mainmatter
\def\ECCV12SubNumber{***}  % Insert your submission number here

\title{Design of Blur Invariant Fiducial for Low Cost Quadcopter} % Replace
% with your title

\titlerunning{ECCV-12 submission ID \ECCV12SubNumber}

\authorrunning{ECCV-12 submission ID \ECCV12SubNumber}

\author{Anonymous ECCV submission}
\institute{Paper ID \ECCV12SubNumber}

\maketitle

\begin{abstract}
Fiducial markers are commonly used to track an object in an unknown environment
and finds use in various applications in Virtual Reality, Medical imaging,
Surveys, etc. The performance of popular fiducials is satisfactory when there is
little motion or no motion in the device obtaining the imagery. But when there
is continuous and swift motion, as in the case of low cost quadcopters,
performance of these fiducials degrade significantly due to motion blur.
Inspired from Circular Data Matrix\cite{NaimarkF02} we have designed a fiducial
that may be thought of as a binary code.  It contains concentric white rings of
equal widths on a black background with a blurred border. Our
algorithm is based on the fact that there is no blur in the direction
perpendicular to the direction of the motion.

\end{abstract}

\section{Introduction}
A fiducial marker or simply a fiducial is an synthetic object placed in the
scene, which can provide additional information about the environment.

Low cost quadcopters such as AR Drone are very unstable which causes non-uniform
motion resulting lot of motion blur in captured images. Also, as image transfer
is done through wireless media using UDP, there is possibility of missing 
intermediate frames . It results in lower frame rate (maximum 20-25 frames per
second (FPS)) instead of normal rate of 30 FPS. This dropping of frames may
cause drastic change in position of object in successive frames. Thus,
performance of traditional tracking methods is not satisfactory for tracking
through quadcopters. Our aim is to design a fiducial which we will be able to track under significant
amount of motion blur and is robust in terms of drastic change in its position
in successive frames.

We observed that blur in a single frame captured through quadcopter is linear.
There is no blur in the direction perpendicular to the direction of motion. We
tried to design a fiducial whose ``signature'' remains intact in any direction.

 
\section{Related Work}
The ARToolkit \cite{ARToolkit02} \cite{kato-artoolkit} is well known toolkit in
AR system, widely used to find pose of the object on which it is placed.
Fiala et.al \cite{Fiala05} developed fiducial named, ARTag, bi-tonal system
containing 2002 planar markers, each consisting of a square border and an
interior region filled with a 6x6 grid of black or white cells. It proved to be
more efficient than \cite{ARToolkit02} in terms of detection rate as well as the
number of different patterns which can be created.  

Concentric Rings\cite{concentric} \\
Multicolor rings \cite{Cho:2001} \cite{Cho97fastcolor} \\
Circular Data Matrix Fiducial System \cite{NaimarkF02} \\

Zhang el. al.\cite{Zhang:2002} and Claus et. al. \cite{ClausF04} have done
quite comprehensive comparative study of various fiducial marker systems with
respect to processing time, recognition rate and accuracy with
respect to viewing angle and distance.

RUNE-Tag \cite{runetag11} \\
Pi-Tag \cite{Pitag13} \\
Blut \cite{Wu:2011}


\section{Design of Fiducial}

\section{Fiducial Detection Algorithm}
\begin{itemize}
  \item Apply Gabor Filter on input image
  \item Find Connected components in Gabor output
  \item Cluster the connected components in bounding box
  \item Detect code in the bounding box
  \begin{itemize}
    \item Run PCA on Gabor output in bounding box
    \item Find intensity profile along first principal component passing through
    centroid
    \item If there are less than four black to white transitions in the
    intensity profile, ignore the bouding box
    \item Otherwise, classify the detected code by the classifer trained on
    standards.
  \end{itemize}
\end{itemize}

\section{Experimental Results}
\subsection{Comparison}
We will compare our results with standard fiducials such as ARTag. Also, we will
compare our results with Blur driven tracker.
\subsubsection{Comparison with ARTag}
\subsubsection{Comparison with Blut} 

\section{Discussion}

\section{Conclusion and Future Work}

Pose Estimation

\bibliographystyle{splncs}
\bibliography{egbib}

\end{document}
