% last updated in April 2002 by Antje Endemann
% Based on CVPR 07 and LNCS, with modifications by DAF, AZ and elle, 2008 and AA, 2010, and CC, 2011

\documentclass[runningheads]{llncs}
\usepackage{graphicx}
\usepackage[font=small]{caption}
\usepackage{subcaption}
\captionsetup{compatibility=false}
\usepackage{amsmath,amssymb} % define this before the line numbering.
\usepackage{ruler}
\usepackage{color}
\usepackage{url}
\usepackage{comment}
\usepackage[width=122mm,left=12mm,paperwidth=146mm,height=193mm,top=12mm,paperheight=217mm]{geometry}
\usepackage{multirow}
\usepackage{tabularx}
\newcolumntype{Y}{>{\centering\arraybackslash}X}
\captionsetup[table]{belowskip=12pt,aboveskip=4pt}
\setlength{\floatsep}{7pt plus 2pt minus 2pt}
\setlength{\textfloatsep}{7pt plus 2pt minus 2pt}
\setlength{\intextsep}{7pt plus 2pt minus 2pt}

\begin{document}
% \renewcommand\thelinenumber{\color[rgb]{0.2,0.5,0.8}\normalfont\sffamily\scriptsize\arabic{linenumber}\color[rgb]{0,0,0}}
% \renewcommand\makeLineNumber {\hss\thelinenumber\ \hspace{6mm} \rlap{\hskip\textwidth\ \hspace{6.5mm}\thelinenumber}}
% \linenumbers \

\pagestyle{headings}
\mainmatter
\def\ECCV14SubNumber{150}  % Insert your submission number here

\title{A Motion Blur Resilient Fiducial Marker For Quadcopter Imaging} % Replace
% with your title

\titlerunning{ECCV-14 submission ID \ECCV14SubNumber}

\authorrunning{ECCV-14 submission ID \ECCV14SubNumber}

\author{Anonymous ECCV submission}
\institute{Paper ID \ECCV14SubNumber}

\maketitle

\begin{abstract}
This paper describes the design and evaluation of a binary fiducial marker
for use with low-cost quadcopters.  Fiducial markers are commonly placed
in environments to provide a uniquely identifiable object in the scene.
For quadcopter applications, fiducials are
often used for evaluating planning algorithms by allowing ground truth
positions in the scene to be detected from the quadcopter's camera. Quadcopters,
however, are subject to quick and unstable motions that can cause significant motion
blur that severely affects the detection rate of existing fiducial markers.
This problem motivated us to design a fiducial that is robust to motion blur.
Our proposed fiducial design uses concentric circles with the observation that
the direction perpendicular to the motion blur direction will be unaffected by
the blur.  This allows the fiducial code along the blur direction to be recognized.
The circular design allows this to work for all motion blur directions.  We
detail the design and detection algorithm for our fiducial and show that our
marker can significantly outperform existing fiducial markers in scenes
captured with a quadcopter.
\keywords{Fiducials, tracking, recognition under blur}
\end{abstract}

\section{Introduction}

The recent availability of low-cost quadcopters has help to fuel significant efforts in
research focused on unmanned aerial vehicles.  Navigation and planning of
these vehicles is typically performed using onboard inertial sensors and/or
vision based modules that uses visual cues in the real world~(e.g.
\cite{Davison:2007,Engel12,Engel13}). However, to evaluate the effectiveness of
navigation methods, fiducial markers are commonly placed into the environment
to provide additional information that serves for ground-truth
positional measurements (e.g.,\cite{Bosnak:2012,Lim09,Klopschitz:2007}).

In order to be effective, fiducial markers (or simply, fiducials) need
to be easily detected in the scene.  A variety of fiducial markers
have been proposed in the literature
(e.g.,~\cite{NaimarkF02,ARToolkit02,Fiala05,Pitag13,runetag11}).
These take the form of binary codes arranged into rectangular grids (\cite{ARToolkit02,Fiala05})
or other geometric primitives arranged in predefined spatial patterns
(\cite{NaimarkF02,Pitag13,runetag11}).
Figure \ref{fig:teaser}-(a) show the popular ARTag\cite{Fiala05} as
seen from a quadcopter.  A problem for existing fiducials is that low-cost quadcopters
often exhibit very quick and erratic physical movements that
results in motion blur in the quadcopter's onboard camera.  This motion blur has
an adverse effect on the  recognition of fiducial  markers.  This can be seen in
Figure \ref{fig:teaser}-(b) where the ARTag cannot be recognized due to motion
blur. This is not too surprising as most  existing fiducials are not explicitly
designed to handle motion blur.

Compounding this problem is the additional issue of dropped video frames from
the quadcopter's wireless communication module.   This means that not only is blurring
introduced, but there may be large discontinuities in the patterns position due
to missing video frames.  As we will show in Section 4, this later problem
makes it challenging to apply tracking algorithms that can exploit temporal
coherence for determining the fiducial positions.

\noindent\textbf{Contribution:}~~To address these problems, we propose a
fiducial that is designed to be robust to motion blur.  Our design is based on
concentric circles as shown in Figure \ref{fig:teaser}-(c,d).  The design is based
on the observation that motion blur from a quadcopter tends to be linear in
nature.  As such, when our code is blurred, there is no blur in the direction
perpendicular to the direction of motion.   This allows the signature of the
fiducial to remain intact in any direction.  In addition, by using concentric
rings, we can treat the presence or absence of a ring as a bit, allowing us to
assign a code to the marker.  Our experiments show that this design can
significantly outperform existing codes in the face of motion blur. As far as
we are aware, this is the first work to propose a blur resistant marker.

The remainder of this paper is organized as follows.   Section 2 gives an
overview of related work related to fiducials as well as the related problem of
tracking.  Section 3 discusses the our design and detection algorithm as well as 
examining the performance of existing codes under motion blur. Section 4 shows several
experiments using quadcopter imagery.  This is followed by a discussion and
summary in Section 5 and 6 respectively.

\begin{figure}
\includegraphics[width=\linewidth]{teaser.pdf}
\caption{This figure shows experimental setup and comparison of
output of ALVAR\cite{alvar} (for detecting ARTag) versus our blur resistant fiducial
in a normal and blurred scene. The scene contains two ARTags and our code.
(a) shows that the ARTag can be detected when the image has no blur.
(b) shows that ARTags are no longer detectable due to the blur.
(c-d) show the same images in (a) and (b) where our code is being detected.}
\label{fig:teaser}
\end{figure}

\section{Related Work}

Our work is related to two areas: fiducial markers and object tracking. We
briefly discuss work done in both areas.

\begin{figure}
 \begin{subfigure}[b]{0.19\textwidth}
  \centering
  \includegraphics[width=\linewidth]{intersense.jpg}
  Intersense\cite{NaimarkF02}
 \end{subfigure}
 \begin{subfigure}[b]{0.14\textwidth}
 \centering
  \includegraphics[width=\linewidth]{pattKanji.pdf}
  ARToolkit\cite{ARToolkit02}
 \end{subfigure}
 \begin{subfigure}[b]{0.14\textwidth}
  \centering
  \includegraphics[width=\linewidth]{ARtag.jpg}
  ARTag\cite{Fiala05}
 \end{subfigure}
 \begin{subfigure}[b]{0.14\textwidth}
  \centering
  \includegraphics[width=\linewidth]{pifiducial.jpg}
  Pi-Tag\cite{Pitag13}
 \end{subfigure}
 \begin{subfigure}[b]{0.14\textwidth}
  \centering
  \includegraphics[width=\linewidth]{our_fiducial}
  Our Fiducial
 \end{subfigure}
 \caption{This figure shows the design of several existing fiducial markers.  While
 each codes have their own pros and cons of various environments, none are designed to be recognized under motion blur.}
 \label{fig:previous_work}
\end{figure}

\noindent{\textbf{Fiducials:}}~Figure~\ref{fig:previous_work} shows several
examples of existing fiducial markers.  Many designs use a two dimensional
barcode inside a rectangular grid. One examples of such a fiducial is from the 
ARToolkit~\cite{ARToolkit02}, a well known toolkit used in many augmented reality (AR) applications. Kato et al.\cite{kato-artoolkit} demonstrated the use of ARToolkit
to find the pose in video based AR conferencing system. 

Fiala et al. \cite{Fiala05} proposed a fiducial termed, ARTag, which is a
bi-tonal system consisting of a square border and an interior 6$\times$6 grid of black or white cells. The improvement in ARTag compared to
ARToolkit was detection of corners instead of detection of lines to find
possible pattern. This proved to be more efficient than \cite{ARToolkit02} in
terms of marker recognition rate as well as the number of different patterns
which can be created.   The reliance on both line and corner detection hampers recognition under motion blur.

There were also attempts to use circular patterns instead of rectangular. 
The use of concentric rings was used  by Gatrell et al.\cite{concentric}
for monocular pose estimation as well as object identification in space. Cho et al.
\cite{Cho:2001,Cho97fastcolor} have used multicolor rings instead of
black and white rings\cite{concentric} to increase possible number of fiducials.
These multicolor rings are used in wide area tracking in large scale
applications.  While based on a concentric rings, these particular approaches require 
the full ring to be recognized which is not possible when the pattern undergoes directional
motion blur.

Naimark et al.~\cite{NaimarkF02} proposed a circular bar code
called the, Circular Data Matrix that is beneficial in terms of
easy detection and ability to have a large number
of uniquely identifiable codes.  Addressing the issue of occlusion, 
Bergamasco et al. \cite{runetag11} proposed the RUNE-tag fiducial divided into number of
circular dots arranged them in circular fashion.
RUNE-tags can be detected even when up to 50\% of fiducial area is occluded.
Bergamasco et al.~\cite{Pitag13} proposed the Pi-Tag fiducial also composed 
of circular structures but arranged in a rectangle to exploit projective invariant cross-ration.
This provided similar occlusion resistance as RUNE-tag but with even less
circular dots. All of these techniques however relies on generic features
(e.g., circle detection) that break down under motion blur.

%Zhang et al.\cite{Zhang:2002} and Claus et al. \cite{ClausF04} have done
%quite comprehensive comparative study of various fiducial marker systems with
%respect to processing time, recognition rate and accuracy with
%respect to viewing angle and distance.

\noindent{\textbf{Tracking:}}~~Fiducial detection between successive video
frames can be considered a tracking problem where the tracked object is fiducial itself.
There is a very large body of research dedicated to tracking and interested
readers are referred to~\cite{Yilmaz:2006} for a survey.

Most tracking methods(~\cite{Ross:2008,Wu:2009,Perez02,Mei:2009} ) assume the
image sequence to be blur free. In reality, however, motion blur in video
sequences is often unavoidable. To this end, Wu et al.\cite{Wu:2011} proposed the
Blur-driven Tracker (BLUT) framework for tracking motion-blurred targets. BLUT
is based on the observation that although motion blurs degrade the visual
features of the target, they at the same time, provide useful cues about the
movements to help tracking.

The BLUT framework successfully tracks blurred target when there is uniform motion
and the position of tracked object does not change drastically in successive
frames. But as we will demonstrate in Section 4, the erratic motion from quadcopters as
well as the problem of dropped video frames is beyond the current ability of such trackers.

\section{Design of Blur Resistant Fiducial}

We begin by first motivating the need for a new blur resistant fiducial by examining
the performance of prior fiducials under motion blur.  After this, we detail
our design as well as the detection algorithm used to find our marker in an image.

\subsection{Examining Prior Fiducials Under Motion Blur}\label{sec:blurtest}

Before we detail the design of our fiducial, we examine the performance of two
popular fiducials under motion blur.  Specifically we examine ARTags
\cite{Fiala05} and PiTags\cite{Pitag13} given their difference in geometry design.

To simulate the appearance of the maker as seen by a quadcopter,  we scaled down
the ARTag and PiTag fiducials to be size 150$\times$150 pixels. Both fiducials are then blurred
using linear motion blur at various orientations with different blurs scales.  
The blur motion ranged from 15 to 50 in magnitude (denoted in pixels),
representing small to significant motion blur. Figure \ref{fig:artag_pitag}
shows the visual appearance of the blurred tags.  We then try to detect the
markers using the ALVAR library~\cite{alvar} and Pi-Tag
library~\cite{ros_pitag}. The table in the left side of
Figure~\ref{fig:artag_pitag} shows the recognition rate (in percentage) of two
fiducial markers at various blur scales over all the different orientation.  
As we can see, the PiTag performance quickly diminished under small amounts of
blur, however, at 35 motion blur, ARTag is recognition rate drops to less than 20\%.

\noindent\begin{minipage}[h!]{\textwidth}
\noindent\begin{minipage}{0.6\textwidth}
\includegraphics[width=\textwidth]{artag_pitag.pdf}
%\captionof{figure}{Blurred AR Tag and Pi--tag with various blur scales}
\end{minipage}
\begin{minipage}{0.35\textwidth}
%\captionof{table}{Recognition rate of AR Tag and Pi-tag fiducial marker atvarious blur scales}
\begin{tabularx}{\textwidth}{|Y|Y|Y|}
\cline{1-3}
\small{Blur} & \multicolumn{2}{c|}{ \small{Recognition Rate}}
\\\cline{2-3}
\small{Scale}& \small{AR Tag} &	\small{Pi-Tag} \\ \cline{1-3}
\small{15} & \small{100} & \small{100} \\ %\cline{1-3}
\small{30} & \small{0} & \small{100} \\  %\cline{1-3}
\small{35} & \small{0} & \small{19} \\ %\cline{1-3}
\small{50} & \small{0} & \small{0} \\ \cline{1-3}
\end{tabularx}
\end{minipage}
\label{fig:artag_pitag}
\captionof{figure}{Figure showing ARTag and Pi-tag blurred with various blur
scales at different orientation. Table on the right shows the recognition rate
(in percent) of both fiducial markers at various blur scales along all blur
orientations.  Recognition rates for both tag is significantly reduced due to
blur. For severe blur, detection is not possible.}
\end{minipage}

\subsection{Blur Resistant Fiducial}

We have designed binary coded fiducial that uses concentric white rings of
equal widths on a black background with a blurred border\footnote{This design
can be easily inverted to have a white background with white rings}. The
outermost and innermost rings represent the start and end of the code and is
embedded in the fiducial.  The binary code is represented by the presence (or
absence) of rings between ``marker'' rings.

\begin{figure}
\centering
  \includegraphics[width=.18\linewidth]{newconcentric_00.pdf}
  \includegraphics[width=.18\linewidth]{newconcentric_01.pdf}
  \includegraphics[width=.18\linewidth]{newconcentric_10.pdf}
  \includegraphics[width=.18\linewidth]{newconcentric_11.pdf}
  \caption{Two bit binary coded fiducials (from left to right: binary code 00,
  binary code 01, binary code 10, binary code 11.)}
  \label{fig:fiducials}
\end{figure}

Depending on which ring is present or absent, the resulting binary code will
change. The number of different patterns depends on the number of bits in the
binary code. For example, if the binary code has three bits, there will be a
maximum of three rings between marker-rings and we end up with eight
different patterns. Figure \ref{fig:fiducials} shows a two bit binary coded
fiducials.

\subsection{Fiducial Detection Algorithm}

\begin{figure}
\centering
\includegraphics[width=\linewidth]{blur_direction.pdf}
\caption{Figure showing how change in blur direction changes the location of
unblurred linear pattern.}
\label{fig:blur_direction}
\end{figure}

Our fiducial detection strategy is different from \cite{NaimarkF02,Pitag13} and works
under significant amounts of blur.   As previously mentioned, our approach works
under the observation that the motion blur for the quadcopter's camera
can be well modelled as linear motion.  This  linear motion blur assumption has been shown to be
reasonable in prior works targeting camera motion blur (e.g.~\cite{Moshe:2003,Moshe:2004}).
Under this assumption, the scene content perpendicular to the blur direction is
unaffected by the blur.  Because of our circular design, the
direction perpendicular to the linear motion will appear as an unblurred linear
pattern.  Figure \ref{fig:blur_direction} shows examples of this using a
pattern  with various motion directions.

Figure \ref{fig:overall_flow} shows the process involved in fiducial
detection process. We give a brief overview of our algorithm here and details
to each step in the following.  Our detection algorithm has four steps. Step 1,
we apply a Gabor filter on the image captured through AR Drone to isolate the
potential locations of the pattern.  Step 2, we find the clusters of patches
found in Gabor output using spatial clustering.  Step 3, performs Principal
Component Analysis (PCA) on each cluster to find the dominate direction
unaffected by blur.  Step 4, based on the direction we extract the intensity
profile of the pattern and classify the code using a K-Nearest neighbor
classifier based on training data.

\begin{figure}[ht!]
\includegraphics[width=\linewidth]{overall_flow.pdf}
\caption{An overview of our algorithm to detect and classify fiducial. 
The four step process includes (Step 1) Gabor filtering,
(Step 2) component clustering, (Step 3) PCA to determine the dominate direction, 
and (Step 4) KNN classification using training-data captured for each pattern.}
\label{fig:overall_flow}
\end{figure}

\noindent Details of each of the steps are as follows:\\
\noindent\textbf{Gabor filter}: A 2D Gabor filter is a Gaussian kernel function
modulated by a sinusoidal plane wave~\cite{Kruizinga:2002}. It is used to find
high gradient patches from the image. In our case, it is used to detect blur
invariant sections of the fiducials. We have applied Gabor filter for eight
different orientations ($\theta = 0, 45, 90, \ldots ,225, 270, 315$).  The
following parameters where used for creating each Gabor kernel: $\lambda$ (wavelength) $= 8$, $\gamma$
(aspect ratio) $= 0.5$, $\sigma$ (spread) $= 0.56\lambda$, $\psi$
(phase angle) $= 0$ (for real part), $\pi/2$ (for imaginary part).
Then L2 norm of outputs along all orientations is calculated and finally L2
normalized image is thresholded with threshold set to $0.4$ (on scale of zero to
one).

\noindent\textbf{Clustering}: The binarized Gabor filter response are
treated as a set of connected components in the image.   This clustering
stage is to find components that are located in a close spatial region.  We do
this by using hierarchical fashion clustering~\cite{ALGLIB}. We used unweighted
average linkage with distance threshold set to 150 for merging clusters in
hierarchical fashion.

\noindent\textbf{PCA}: For each clustered set of components, we apply 
PCA to determine the dominate direction.  This is done by examining the
orientation of the first principal component.  We then extract an intensity
profile patch in the input image along this direction as it extends through the
bounding box of the cluster. The signature of this profile will be used to
identify the code.

\noindent\textbf{Classification}: As mentioned above, a small image patch containing
the intensity profile of the fiducial is used to identify the code.   We found
that a training-based method using KNN gave better results than trying to directly determine
the number of transitions in the image patch.  To build the training-data, a
synthetic fiducial pattern is blurred along 36 orientations (0, 10, 20, \ldots
, 350) and its intensity profile along the first principal component from every
output is taken as training data for that fiducial pattern. Figure
\ref{fig:training_data} shows the process of creating this training data for
``01'' fiducial with.  For an input image patch, we normalize its intensity
range and then compare it against the training-data. The class label from the
closest top $K=5$ images in the training-data is used to label the patch.

\begin{figure}[h!]
\centering
  \includegraphics[width=\linewidth]{training_data.pdf}
  \caption{Process to create training data for the fiducial pattern with binary
  code ``01''. The synthetic pattern is blurred along various orientations. Gabor
  filter is applied on the blurred pattern. Intensity profile is found along the
  first principal component of the clustered Gabor output. Same process is used
  to create training data for other fiducial patterns.}
  \label{fig:training_data}
\end{figure}

To increase the classification accuracy, training data for patterns having same
number of rings is grouped together; e.g., in two bit binary coded fiducial,
training data for pattern ``01'' and ``10'' will be grouped together, In three
bit binary coded fiducial, training data for pattern ``001'', ``010'' and
``100'' will form one group while training data for pattern ``110'', ``011'' and
``101'' will be in other group, etc. Now, depending on the number of detected
rings in test pattern, it is matched against corresponding group of training data, again
using  K-Nearest Neighbor Classifier. In this way, if we detect either zero
rings or maximum possible rings in the test pattern, there will be no need to do
further classification.

\section{Experimental Validation}

We have implemented our algorithm in C++ using OpenCV library.
Experiments were performed on a PC with Intel Core i7 processor(@3.4GHz) and 4GB RAM.
Source code and data sets in this paper will be made publicly available.  Video clips
have also been provided in the supplemental materials.

Our system has been tested on several image sequences captured from an AR Drone
quadcopter.  The Drone was flown indoors looking at patterns attached to various walls.
Each image sequence contains frames containing different fiducial
patterns. Sample output for each fiducial pattern is shown in Figure
\ref{fig:output0} -- Figure \ref{fig:output3}. Our detection process takes
around 0.3 seconds which translates to around three to four frames per second.



Our system has also been tested on images containing multiple fiducial patterns
in the same frame. Our algorithm successfully detected all fiducial patterns as
well as correctly classified them as shown in Figure \ref{fig:output_all}.

\begin{figure}
\centering
  \includegraphics[width=.45\linewidth]{output_all_2.jpg}
  \includegraphics[width=.45\linewidth]{new_results/output_test_all1.jpg}
  \caption{Sample outputs containing all two bit binary coded fiducial patterns
  in single frame}
  \label{fig:output_all}
\end{figure}

\subsection{Comparisons}

We compare our results with the commonly used ARTag. We also compare 
our results with Blur-driven tracker (BLUT)\cite{Wu:2011}.
\subsubsection{Comparison with ARTag}
First, we have repeated blur simulation experiment on our fiducials as reported
in Section~\ref{sec:blurtest}.  Specifically, we build our blur resistant
patterns size 150x150 and blurred them along various orientations with different
blur scales. Then, we tried to detect our fiducial using a 2-bit patterns using
algorithm presented in earlier section. 

\begin{figure}
\includegraphics[width=\linewidth]{blur_maximum.pdf}
\caption{Result of our fiducial detection algorithm on blur simulated data}
\label{fig:blur_maximum}
\end{figure}

The comparison of recognition rate is shown in Figure
\ref{fig:recognition_rate}. In this experiment, we use even more blur, up to 65
pixels.  The graph shows in Figure \ref{fig:recognition_rate} that our
recognition to 100\% for all codes except the ``00'' which has is undetectable
after 50 pixels blur (which is still significantly better than ARTag).  Reasons
for the ``00'' tags lower performance is discussed in Section~\ref{sec:discussion}.

We have also checked the difference between the actual center of the fiducial
and center found by our algorithm, on blur simulated data along different blur
orientations. From Table \ref{tab:blur_angle_center}, it can be clearly seen
that the difference between the actual center and center found by algorithm is
less than three percent of the diameter of fiducial. 

\noindent\begin{minipage}{\linewidth}
\noindent\begin{minipage}{0.67\linewidth}
\includegraphics[width=\linewidth]{recognition_rate.png}
\captionof{figure}{Comparison of recognition rate of AR Tag and our fiducials on
blur simulated data at various blur scales. It can be clearly seen that except
fiducial with binary code ``00'', our fiducial patterns are
recognised all times.}
\label{fig:recognition_rate}
\end{minipage}
\noindent\begin{minipage}{0.32\linewidth}
\begin{tabularx}{\textwidth}{|c|Y|Y|}
\cline{1-3}
\small{Blur} & \small{Actual} & \small{Found} \\
\small{Angle} & \small{center} & \small{center} \\
\cline{1-3}
0 & (55,75) & (54,75) \\
22 & (56,67) & (55,64) \\
45 & (61,61) & (59,62) \\
67 & (67,56) & (63,56) \\
90 & (75,55) & (75,53) \\  \cline{1-3}
\end{tabularx}
\captionof{table}{Table showing the actual center of our fiducial and center
found by our algorithm, when blurred along various blur angles (Blur Scale =
40).}
\label{tab:blur_angle_center}
\end{minipage}
\end{minipage}

We also compared results on the recorded feed using AR Drone quadcopter. In our
experimental setup, we have placed two AR Tags in the scene along with our fiducial  to
compare the resilience of blur by each fiducial type. We have used
ar\_track\_alvar, ROS Wrapper for ALVAR library \cite{ros_alvar}, to detect AR
Tags from the stream captured through quadcopter. In each test dataset, we used
different two bit binary coded fiducial and recorded video of around two minute
duration (i.e., around 1000 frames).   The quadcopter was flown in a routine
manner about the room with its camera facing the wall (see supplemental
videos). The comparison of recognition rate is shown in Table
\ref{tab:recongition_accuracy}. Classification accuracy of all fiducials (ARTag
as well as ours) ranges from 86.5\% to 94.1\%, while the AR Tag is 60.3\% to 65.6\%.
\begin{figure}
\begin{subfigure}{\textwidth}
\centering
  \includegraphics[width=0.48\linewidth]{output_00.jpg}
  \includegraphics[width=0.48\linewidth]{new_results/output_00.jpg}
  \caption{Binary code ``00''}
  \label{fig:output0}
\end{subfigure}
\begin{subfigure}{\textwidth}
\centering
  \includegraphics[width=0.48\linewidth]{output_01.jpg}
  \includegraphics[width=0.48\linewidth]{new_results/output_01.jpg}
  \caption{Binary code ``01''}
  \label{fig:output1}
\end{subfigure}
\begin{subfigure}{\textwidth}
\centering
  \includegraphics[width=0.48\linewidth]{output_10.jpg}
  \includegraphics[width=0.48\linewidth]{new_results/output_10.jpg}
  \caption{Binary code ``10''}
  \label{fig:output2}
\end{subfigure}
\begin{subfigure}{\textwidth}
\centering
  \includegraphics[width=0.48\linewidth]{output_11.jpg}
  \includegraphics[width=0.48\linewidth]{new_results/output_11.jpg}
  \caption{Binary code ``11''}
  \label{fig:output3}
  \end{subfigure}
  \caption{Sample outputs for two bit binary coded fiducials on different
  datasets. Class label attached to the detected fiducial is decimal equivalent
  of the binary code embedded in the detected fiducial.}
\end{figure}

\begin{table}[h!]
\caption{Comparison of recognition rate of AR Tag and our fiducials on real
data captured through AR Drone. Each row shows analysis of a test
dataset captured for our fiducial with different binary code embedded in it.
Each dataset has around 1000 frames captured in around two minutes.}
\centering
\begin{tabularx}{\textwidth}{|c|Y|Y|Y|Y|}
\cline{1-5}
\multirow{2}{*}{Test \#} & \multirow{2}{*}{Number of Frames}
&\multirow{2}{*}{Binary Code} &\multicolumn{2}{c|}{Recognition Rate (\%)} \\
\cline{4-5} & & & Our Fiducial & AR Tag\\\cline{1-5}
1 & 1205 & 00 & 86.5 &  65.6 \\ \cline{1-5}
2 & 1047 & 01 & 94.1 &  61.9 \\ \cline{1-5}
3 & 1102 & 10 &  92.74 & 62.4 \\ \cline{1-5}
4 & 1081 & 11 & 93.54 & 60.3 \\ \cline{1-5}
\end{tabularx}
\label{tab:recongition_accuracy}
\end{table}

\subsubsection{Comparison with BLUT:}

We also compare out approach with tracking designed for blurred input scenes. 
We have used four image sequences (consisting of around 1000 frames each), each
one containing a different blur resistent fiducial pattern to test the performance of
BLUT\cite{Wu:2011} and compare it with our result on the same image sequence.
From top rows of Figure \ref{fig:BLUT_compare_00}-\ref{fig:BLUT_compare_11}. From
the images we can see that BLUT is able to track the fiducial when the position of
fiducial does not change too much in successive frames. Also, it can be seen
that, once BLUT looses the track of the fiducial, it is not able to recover. 
Since our approach detects the code at each frame, large changes in the patterns
position is not an issue.   Our detection results are also shown in Figure
\ref{fig:BLUT_compare_00}-\ref{fig:BLUT_compare_11}.

We have also found that, even if we reset the BLUT tracker after loosing track,
it will loose track again after around 100 frames i.e., approximately within 6
seconds. When we checked the timestamp data from image header captured through
AR Drone, we found that, there was difference of 0.14 seconds between two
successive frames in the earlier case, which clearly suggests the dropping of
frame (e.g., normal 30fps should be 0.33 seconds). Also,  there were around 50
instances in 1000 frames where timestamp difference between two successive
frames was greater than 0.1 seconds. As such, it appears one of the main
culprits causing the BLUT tracker to fail is the dropping of frames combined
with unstable motion of quadcopter resulting in large discrepancies in the
patterns position between successive frames.

\begin{figure}
\begin{subfigure}[b]{.19\textwidth}
\includegraphics[width=\linewidth]{BLUT_output_00/2.jpg}
\end{subfigure}
\begin{subfigure}[b]{.19\textwidth}
\includegraphics[width=\linewidth]{BLUT_output_00/3.jpg}
\end{subfigure}
\begin{subfigure}[b]{.19\textwidth}
\includegraphics[width=\linewidth]{BLUT_output_00/4.jpg}
\end{subfigure}
\begin{subfigure}[b]{.19\textwidth}
\includegraphics[width=\linewidth]{BLUT_output_00/5.jpg}
\end{subfigure}
\begin{subfigure}[b]{.19\textwidth}
\includegraphics[width=\linewidth]{BLUT_output_00/6.jpg}
\end{subfigure}\\
\begin{subfigure}[b]{.19\textwidth}
\includegraphics[width=\linewidth]{BLUT_input_00/output2.jpg}
\end{subfigure}
\begin{subfigure}[b]{.19\textwidth}
\includegraphics[width=\linewidth]{BLUT_input_00/output3.jpg}
\end{subfigure}
\begin{subfigure}[b]{.19\textwidth}
\includegraphics[width=\linewidth]{BLUT_input_00/output4.jpg}
\end{subfigure}
\begin{subfigure}[b]{.19\textwidth}
\includegraphics[width=\linewidth]{BLUT_input_00/output5.jpg}
\end{subfigure}
\begin{subfigure}[b]{.19\textwidth}
\includegraphics[width=\linewidth]{BLUT_input_00/output6.jpg}
\end{subfigure}
\caption{Top: Output of BLUT\cite{Wu:2011} on sample image sequence containing ``00''
binary coded fiducial. Bottom: Output of our algorithm on the same image
sequence. BLUT is able to track the fiducial till third frame, but from fourth
frame, BLUT looses track of the fiducial. In first three frames, size of the
fiducials is less but in fourth and fifth frame it is bigger, hinting sudden
forward movement of quadcopter.}
\label{fig:BLUT_compare_00}
\end{figure}

\begin{figure}
\begin{subfigure}[b]{.19\textwidth}
\includegraphics[width=\linewidth]{BLUT_output_01/11.jpg}
\end{subfigure}
\begin{subfigure}[b]{.19\textwidth}
\includegraphics[width=\linewidth]{BLUT_output_01/12.jpg}
\end{subfigure}
\begin{subfigure}[b]{.19\textwidth}
\includegraphics[width=\linewidth]{BLUT_output_01/13.jpg}
\end{subfigure}
\begin{subfigure}[b]{.19\textwidth}
\includegraphics[width=\linewidth]{BLUT_output_01/14.jpg}
\end{subfigure}
\begin{subfigure}[b]{.19\textwidth}
\includegraphics[width=\linewidth]{BLUT_output_01/15.jpg}
\end{subfigure}\\
\begin{subfigure}[b]{.19\textwidth}
\includegraphics[width=\linewidth]{BLUT_input_01/output11.jpg}
\end{subfigure}
\begin{subfigure}[b]{.19\textwidth}
\includegraphics[width=\linewidth]{BLUT_input_01/output12.jpg}
\end{subfigure}
\begin{subfigure}[b]{.19\textwidth}
\includegraphics[width=\linewidth]{BLUT_input_01/output13.jpg}
\end{subfigure}
\begin{subfigure}[b]{.19\textwidth}
\includegraphics[width=\linewidth]{BLUT_input_01/output14.jpg}
\end{subfigure}
\begin{subfigure}[b]{.19\textwidth}
\includegraphics[width=\linewidth]{BLUT_input_01/output15.jpg}
\end{subfigure}
\caption{Top: Output of BLUT\cite{Wu:2011} on sample image sequence containing
``01'' binary coded fiducial, Bottom: Output of our algorithm on the same image
sequence. BLUT looses the track from third frame.}
\label{fig:BLUT_compare_01}
\end{figure}

\begin{figure}
\begin{subfigure}[b]{.19\textwidth}
\includegraphics[width=\linewidth]{BLUT_output_10/1.jpg}
\end{subfigure}
\begin{subfigure}[b]{.19\textwidth}
\includegraphics[width=\linewidth]{BLUT_output_10/2.jpg}
\end{subfigure}
\begin{subfigure}[b]{.19\textwidth}
\includegraphics[width=\linewidth]{BLUT_output_10/3.jpg}
\end{subfigure}
\begin{subfigure}[b]{.19\textwidth}
\includegraphics[width=\linewidth]{BLUT_output_10/4.jpg}
\end{subfigure}
\begin{subfigure}[b]{.19\textwidth}
\includegraphics[width=\linewidth]{BLUT_output_10/5.jpg}
\end{subfigure}\\
\begin{subfigure}[b]{.19\textwidth}
\includegraphics[width=\linewidth]{BLUT_input_10/output1.jpg}
\end{subfigure}
\begin{subfigure}[b]{.19\textwidth}
\includegraphics[width=\linewidth]{BLUT_input_10/output2.jpg}
\end{subfigure}
\begin{subfigure}[b]{.19\textwidth}
\includegraphics[width=\linewidth]{BLUT_input_10/output3.jpg}
\end{subfigure}
\begin{subfigure}[b]{.19\textwidth}
\includegraphics[width=\linewidth]{BLUT_input_10/output4.jpg}
\end{subfigure}
\begin{subfigure}[b]{.19\textwidth}
\includegraphics[width=\linewidth]{BLUT_input_10/output5.jpg}
\end{subfigure}
\caption{Top: Output of BLUT\cite{Wu:2011} on sample image sequence containing
``10'' binary coded fiducial, Bottom: Output of our algorithm on the same image
sequence. From second frame itself, BLUT looses the track.}
\label{fig:BLUT_compare_10}
\end{figure}

\begin{figure}
\begin{subfigure}[b]{.19\textwidth}
\includegraphics[width=\linewidth]{BLUT_output_11/2.jpg}
\end{subfigure}
\begin{subfigure}[b]{.19\textwidth}
\includegraphics[width=\linewidth]{BLUT_output_11/3.jpg}
\end{subfigure}
\begin{subfigure}[b]{.19\textwidth}
\includegraphics[width=\linewidth]{BLUT_output_11/4.jpg}
\end{subfigure}
\begin{subfigure}[b]{.19\textwidth}
\includegraphics[width=\linewidth]{BLUT_output_11/5.jpg}
\end{subfigure}
\begin{subfigure}[b]{.19\textwidth}
\includegraphics[width=\linewidth]{BLUT_output_11/6.jpg}
\end{subfigure}\\
\begin{subfigure}[b]{.19\textwidth}
\includegraphics[width=\linewidth]{BLUT_input_11/output2.jpg}
\end{subfigure}
\begin{subfigure}[b]{.19\textwidth}
\includegraphics[width=\linewidth]{BLUT_input_11/output3.jpg}
\end{subfigure}
\begin{subfigure}[b]{.19\textwidth}
\includegraphics[width=\linewidth]{BLUT_input_11/output4.jpg}
\end{subfigure}
\begin{subfigure}[b]{.19\textwidth}
\includegraphics[width=\linewidth]{BLUT_input_11/output5.jpg}
\end{subfigure}
\begin{subfigure}[b]{.19\textwidth}
\includegraphics[width=\linewidth]{BLUT_input_11/output6.jpg}
\end{subfigure}
\caption{Top: Output of BLUT\cite{Wu:2011} on sample image sequence containing
``11'' binary coded fiducial, Bottom: Output of our algorithm on the same image
sequence. BLUT lost the track from third frame. There is sudden reversal of
direction of quadcopter from third frame. In first two frames quadcopter was
going up, but suddenly it moved down in third frame.}
\label{fig:BLUT_compare_11}
\end{figure}

\section{Discussion}\label{sec:discussion}

We have demonstrated the effectiveness of our blur invariant fiducial both
synthetically and on real video clips captured from a quadcopter.   Our approach obtains
approximately 86-95\% recognition rate on fiducals in real scenes compared to existing codes
that are around 64\%.  We discuss some limitations of our approach in the following. 

\noindent\textbf{Processing Time:}~~Currently our processing time (0.3 seconds
per frame) makes it difficult to do the fiducial detection in real-time.  As such, we currently
envision this will be used in an offline manner for performing analysis of flight paths.
Code profiling revealed that the Gabor filtering takes most of the time (0.03
-- 0.04 seconds per orientation). As we are applying Gabor filter along eight
orientations, we should think of some optimized way to do filtering.
Alternatively, some replacement for Gabor filter needs to be find out.

\noindent\textbf{False Negatives:}~~We found that sometimes our detection
algorithm fails to recognize our fiducial with binary code ``00'' when there is
large amount of blur.  When we
investigated the problem, we found that if there is too much blur the Gabor
output the innermost ring response it too low and not detected properly.  
This problem can be resolved by increasing radius of innermost ring to 
reduce the effect of blur on the innermost ring.  

\noindent\textbf{Pose Estimation:}~~We note that other markers are able to give
full pose estimation after detection.  However, we are only able to reliably detect
the center point and therefore cannot estimate pose.  Of course, if four
markers were used in a known order pose could be estimated. The current
resolution of onboard camera will be main hurdle to clear before we can
effectively use multiple markers in each scene.

\noindent\textbf{Number of Fiducials:}~~Currently, compared to AR Tag fiducial
markers, we are able to generate less number of fiducials. Many applications in
robotics (e.g. quadcopter navigation) may not require large number of fiducial
markers as AR Tag. Most of the time, it is sufficient to have 4-6
different fiducial patterns. Still, we may be able to generate more number of
fiducials by using color background instead of black. E.g., if we just use Red,
Green and Blue as background color, number of fiducials will be increased by
factor of three.

\section{Conclusion}

Quadcopters are subject to quick and unstable motions that can cause significant
motion blur in the captured images. It severely affects the detection rate of
existing fiducial markers. We proposed the design of a fiducial that is robust
to motion blur. Our design of contrasting concentric rings is based on the the
observation that the direction perpendicular to the motion blur direction will
be unaffected by the blur and therefore still be recognizable. We have shown
through experimental validation that our fiducial will work under significant
amount of motion blur and can significantly out perform existing fiducial
markers under this scenario.

\bibliographystyle{splncs}
\bibliography{egbib}

\end{document}