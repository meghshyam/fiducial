% last updated in April 2002 by Antje Endemann
% Based on CVPR 07 and LNCS, with modifications by DAF, AZ and elle, 2008 and AA, 2010, and CC, 2011

\documentclass[runningheads]{llncs}
\usepackage{graphicx}
\usepackage{amsmath,amssymb} % define this before the line numbering.
\usepackage{ruler}
\usepackage{color}
\usepackage{url}
\usepackage[width=122mm,left=12mm,paperwidth=146mm,height=193mm,top=12mm,paperheight=217mm]{geometry}
\begin{document}
% \renewcommand\thelinenumber{\color[rgb]{0.2,0.5,0.8}\normalfont\sffamily\scriptsize\arabic{linenumber}\color[rgb]{0,0,0}}
% \renewcommand\makeLineNumber {\hss\thelinenumber\ \hspace{6mm} \rlap{\hskip\textwidth\ \hspace{6.5mm}\thelinenumber}}
% \linenumbers \
\pagestyle{headings}
\mainmatter
\def\ECCV14SubNumber{150}  % Insert your submission number here

\title{Design of Blur Invariant Fiducial for Low Cost Quadcopter} % Replace
% with your title

\titlerunning{ECCV-14 submission ID \ECCV14SubNumber}

\authorrunning{ECCV-14 submission ID \ECCV14SubNumber}

\author{Anonymous ECCV submission}
\institute{Paper ID \ECCV14SubNumber}

\maketitle

\begin{abstract}
Fiducial markers are commonly used to track an object in an unknown environment
and finds use in various applications in Virtual Reality, Medical imaging,
Surveys, etc. The performance of popular fiducials is satisfactory when there is
little motion or no motion in the device obtaining the imagery. But when there
is continuous and swift motion, as in the case of low cost quadcopters,
performance of these fiducials degrade significantly due to motion blur.
Inspired from Circular Data Matrix\cite{NaimarkF02} we have designed a fiducial
that may be thought of as a binary code.  It contains concentric white rings of
equal widths on a black background with a blurred border. Our
algorithm is based on the fact that there is no blur in the direction
perpendicular to the direction of the motion.
\dots
\keywords{Fiducials, Blur Invariant Tracking,}
\end{abstract}

\section{Introduction}
A fiducial marker or simply a fiducial is an synthetic object placed in the
scene, which can provide additional information about the environment.

Low cost quadcopters such as AR Drone are very unstable which causes non-uniform
motion resulting lot of motion blur in captured images. Also, as image transfer
is done through wireless media using UDP, there is possibility of missing 
intermediate frames . It results in lower frame rate (maximum 20-25 frames per
second (FPS)) instead of normal rate of 30 FPS. This dropping of frames may
cause drastic change in position of object in successive frames. Thus,
performance of traditional tracking methods is not satisfactory for tracking
through quadcopters. Our aim is to design a fiducial which we will be able to
track under significant amount of motion blur and is robust in terms of drastic
change in its position in successive frames.

We observed that blur in a single frame captured through quadcopter is linear.
There is no blur in the direction perpendicular to the direction of motion. We
tried to design a fiducial whose ``signature'' remains intact in any direction.

 
\section{Related Work}
The ARToolkit \cite{ARToolkit02} \cite{kato-artoolkit} is well known toolkit in
AR system, widely used to find pose of the object on which it is placed.
Fiala et.al \cite{Fiala05} developed fiducial named, ARTag, bi-tonal system
containing 2002 planar markers, each consisting of a square border and an
interior region filled with a 6x6 grid of black or white cells. It proved to be
more efficient than \cite{ARToolkit02} in terms of detection rate as well as the
number of different patterns which can be created.  

Concentric Rings\cite{NaimarkF02} \\
Multicolor rings \cite{Cho:2001} \cite{Cho97fastcolor} \\
Circular Data Matrix Fiducial System \cite{NaimarkF02} \\

Zhang el. al.\cite{Zhang:2002} and Claus et. al. \cite{ClausF04} have done
quite comprehensive comparative study of various fiducial marker systems with
respect to processing time, recognition rate and accuracy with
respect to viewing angle and distance.

RUNE-Tag \cite{runetag11} \\
Pi-Tag \cite{Pitag13} \\
Blut \cite{Wu:2011}


\section{Design of Fiducial}
Inspired from Circular Data Matrix\cite{NaimarkF02} we have designed a fiducial
that may be thought of as a binary code.  It contains concentric white rings of
equal widths on a black background with a blurred border. The outermost and
innermost rings represent the start and end of the code and is embedded in the
fiducial; these are not considered part of the code itself. The binary code is
represented by the presence (or absence) of rings between ``marker'' rings.
Depending on which ring is present or absent, the resulting binary code will
change. The number of different patterns depends on the number of bits in the
binary code. For example, if the binary code has three bits, there will be a
maximum of three rings between``marker'' rings and we end up with eight
different patterns.

\begin{figure}
\centering
  \includegraphics[width=.2\linewidth]{newconcentric_00.pdf}  
  \includegraphics[width=.2\linewidth]{newconcentric_01.pdf}
  \includegraphics[width=.2\linewidth]{newconcentric_10.pdf}
  \includegraphics[width=.2\linewidth]{newconcentric_11.pdf}
  \caption{Two bit binary coded Fiducials (from left to right: Binary Code 00,
  Binary Code 01, Binary Code 10, Binary Code 11)}
  \label{fig:fiducials}
\end{figure}

\section{Fiducial Detection Algorithm}
Overall flow of fiducial detection algorithm can be summarised as follows:
\begin{itemize}
  \item Apply Gabor Filter on input image
  \item Find Connected components in Gabor output
  \item Cluster the connected components in bounding box
  \item Detect code in the bounding box
  \begin{itemize}
    \item Run PCA on Gabor output in bounding box
    \item Find intensity profile along first principal component passing through
    centroid
    \item If there are less than four black to white transitions in the
    intensity profile, ignore the bouding box
    \item Otherwise, classify the detected code by the classifer trained on
    standards.
  \end{itemize}
\end{itemize}

\textbf{Gabor Filter}: A 2D Gabor filter is a Gaussian kernel function modulated
by a sinusoidal plane wave. It is used to find high gradient patches from the
image. In our case, it will detect blur invariant sections of the fiducials. We
have applied Gabor filter for eight different orientations and then calculated
L2 norm of all outputs.

\textbf{Clustering}: Connected components are found from the Gabor output. 
These connected components are then clustered in hierarchical fashion, to find
the group of closely positioned patches.

\textbf{PCA}: Principal Component Analysis is done on the clustered Gabor
output. Along 1st principal component, intensity profile in the original image is found.
Number of transitions in the intensity profile will give us the number of rings.

\textbf{Classification}: Each synthetic fiducial pattern is blurred along 36
orientations (0, 10, 20, \ldots , 350) and intensity profile along first
principal component from every output is taken as training data for that fiducial pattern.
Training data for patterns having same number of rings is grouped together;
e.g., in two bit binary coded fiducial, training data for pattern ``01'' and
``10'' will be grouped together. Now, depending on the number of detected
rings, test pattern is matched against corresponding training data using
K-Nearest Neighbor Classifier (K=5).

\begin{figure}
\centering
  \includegraphics[width=.3\linewidth]{concentric_01_5_blurred_180.jpg}  
  \includegraphics[width=.3\linewidth]{concentric_01_5_blurred_45.jpg}
  \includegraphics[width=.3\linewidth]{concentric_01_5_blurred_90.jpg}  
  \caption{Sample images used to create training data for the fiducial pattern
  with binary code ``01''}
\end{figure}

\section{Experimental Validation}
Our system has been tested on the image sequences captured from AR Drone
quadcopter. Each image sequence contains frames containing different fiducial
pattern. Sample output for each fiducial pattern is shown in Fig.
\ref{fig:output0} -- Fig. \ref{fig:output3}.

\begin{figure}
\centering
  \includegraphics[width=.8\linewidth]{output_00.jpg}
  \caption{Sample output for the fiducial pattern with binary code ``00''}
  \label{fig:output0}
\end{figure}

\begin{figure}
\centering
  \includegraphics[width=.8\linewidth]{output_01.jpg}
  \caption{Sample output for the fiducial pattern with binary code ``01''}
  \label{fig:output1}
\end{figure}

\begin{figure}
\centering
  \includegraphics[width=.8\linewidth]{output_10.jpg}
  \caption{Sample output for the fiducial pattern with binary code ``10''}
  \label{fig:output2}
\end{figure}

\begin{figure}
\centering
  \includegraphics[width=.8\linewidth]{output_11.jpg}
  \caption{Sample output for the fiducial pattern with binary code ``11''}
  \label{fig:output3}
\end{figure}

Our system has also been tested on images containing multiple fiducial patterns
in the same frame. Our algorithm successfully detected all fiducial patterns as
well as correctly classified them as shown in Fig. \ref{fig:output_all}
\begin{figure}
\centering
  \includegraphics[width=.8\linewidth]{output_all_2.jpg}
  \caption{Sample output containing all two bit binary coded fiducial patterns}
  \label{fig:output_all}
\end{figure}

\subsection{Comparison}
We will compare our results with standard fiducials such as ARTag. Also, we will
compare our results with Blur driven tracker.
\subsubsection{Comparison with ARTag}
\subsubsection{Comparison with Blut} 

\section{Discussion}

\section{Conclusion and Future Work}

Pose Estimation

\bibliographystyle{splncs}
\bibliography{egbib}

\end{document}
